\chapter{Összegzés}
\label{ch:sum}

Szakdolgozatomban egy Python kódok átalakítására alkalmas szoftvert mutattam be.
Az átalakításokat végző szoftver segítségével GitHub-ról bányászott kódokból generáltam
ekvivalens és nem ekvivalens forráskód-párokat tartalmazó adathalmazt.
Az adathalmazt egy forráskód-pár ekvivalenciáját eldöntő mélytanuló neuronháló tanítására használhatjuk.

Az átalakítások szemléltetésére grafikus felhasználói felületű desktop alkalmazást is készítettem,
amivel a felhasználó kipróbálhatja az átalakításokat.

A szoftver forráskódja többnyire objektum orientált,
a megvalósítás során a bővíthetőségre törekedtem.
A jövőben a szoftvert több rétegét is érdemes lehet bővíteni,
például a meglévő interfészeket használva új átalakításokat adhatunk a szoftverhez és
a GUI alkalmazás nézeteit is könnyen lecserélhetjük az MVC architektúrának köszönhetően.

Érdemes lehet változtatni az adathalmaz generálásán is. 
A szabály alapú AST-szintű átalakítások nem tudnak nagy szintaktikai változtatásokat végrehajtani,
így a bemeneti és az átalakított kódok közötti különbség sokszor kicsi.
Ezért a neuronhálót tanító adathalmazt érdemes lehet kiegészíteni egy létező,
szemantikusan ekvivalens, de szintaktikailag különböző kódokat 
tartalmazó adathalmazzal és az azon végzett átalakításokkal (pl. project-codenet).
