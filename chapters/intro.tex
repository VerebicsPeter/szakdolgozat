\chapter{Bevezetés}
\label{ch:intro}

% TODO: CIKKEK

Szakdolgozatom témája Python forráskódok átalakítása, és ezen átalakítások szemléltetése.

A motiváció az átalakítások mögött egy olyan adathalmaz generálása amiben
ekvivalenciával felcímkézett forráskód-párok szerepelnek.
Egy ilyen adathalmazt felhasználhatunk egy mélytanuló
neuronháló tanítására, ami forráskód-párok ekvivalenciáját dönti el.

Az ekvivalencia eldöntése fontos feladat, mivel egyre több, kódokat gépi tanulással refaktoráló eszköz létezik.
Ezek az eszközök egy kód refaktorálására adott promt hatására sokszor a kód jelentését is megváltoztatják,
ami nem felel meg a refaktorálás definíciójának
(egy átalakítás akkor refaktorálás, ha szemantikusan ekvivalens az átalakított kóddal).

Egy ekivalenciát eldöntő neuronháló képes lenne kiszűrni a szemantikusan nem ekvivalens átalakításokat,
ezzel javítva a gépi tanulást használó refaktoráló eszközök hatékonyságán.
Tehát ekvivalens és nem ekvivalens kódokat generálva felépíthetünk egy adathalmazt, ami
ekvivalenciával felcímkézett kódpárokat tartalmaz, és alkalmas egy fent leírt neuronháló
tanítására.

Az általam implementált átalakítások absztrakt szintaxisfák (AST-k) módosításával működnek.
Egy forráskód fordítása közben a szintaktikus elemző előállítja a kód AST-jét,
ami a kódot egy fa adatstruktúrával reprezentálja.
Az AST-nek a szemantikus elemzésben van szerepe,
de használhatjuk kódok átalakítására is, mivel visszaalakítható forráskóddá.

Az általam megvalósított átalakítások a Python \emph{ast} modult \cite{pythonAST} használják,
amely része a Python standard könyvtárának.

Az átalakításokat elvégzéséhez szabályokat definiáltam.
Átalakításkor a Python kódból létrehozott AST-n végrehajtunk egy szabályt alkalmazó függvényt.
Ez a függvény a szabály segítségével megváltoztatja az AST-t, amit ha visszaalakítunk forráskóddá,
egy megváltozott Python kódot kapunk.

A szakdolgozatomban ekvivalens és nem ekvivalens átalakításokhoz is definiálok szabályokat.
Egy átalakítás akkor tekinthető ekvivalensnek, ha a kód szemantikáját nem változtatja meg.
Például a Python-ban is teljesül a számok körében a szorzás kommutatív tulajdonsága.
Tehát, ha egy Python kódban két szám szorzásánál a bal és jobb operandust megcseréljük,
akkor a szorzás eredménye nem változik, vagyis ez az átalakítás ekvivalens.

Az adathalmazban az ekvivalens kódok generálásához saját szabályok mellett
a \emph{ruff} Python linter és formatter
\cite{Ruff} szabályait is felhasználtam.
A \emph{ruff} már létező Python lintereket implementál Rust programozási nyelven,
így sok más Python refaktoráló eszköz szabályait is képes elvégezni,
amelyek tökéletesek az általam implementált szabályok kiegészítésére.

A szakdolgozatom következő fejezeteiben az adathalmaz generálására és az átalakítások
szemléltetésére alkalmas szoftver használatát és működését részletezem.

% TODO
% Szerintem jó lenne pár cikkre hivatkozni, és így a bibliográfia is nőne.
% Ilyen stílusban, hogy:
% "A hasonló vagy ekvivalens működésű forráskódok fellelése, vagy az ekvivalencia megállapítása egy sokat kutatott terület \cite{...}. XY et al. például ... stb.
