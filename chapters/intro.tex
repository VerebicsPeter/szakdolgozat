\chapter{Bevezetés}
\label{ch:intro}

Egyre elterjedtebbek a forráskódokat mély tanuláson alapuló módszerekkel refaktoráló szoftverek. Ezek a szoftverek sokszor helytelenül refaktorálják a bemeneti forráskódot, azaz változtatnak a kód jelentésén. A forráskódok ekvivalenciájának vizsgálata fontos feladat, ugyanis ha lenne lehetőség egy forráskód pár ekvivalenciájának megállapítására, akkor a refaktoráló szoftverek által adott hibás megoldásokat ki lehetne szűrni.

Az ekvivalencia meghatározásának feladata megoldható mély tanuláson alapuló módszer segítségével, ami a párba állított forráskódokról eldönti, hogy azok ekvivalensek-e. Egy ilyen háló tanításához szükség van egy adathalmazra, ami forráskód párokat tartalmaz felcímkézve azzal, hogy ekvivalensek-e. A szakdolgozatomban Python kódok ekvivalenciáját eldöntő, mélytanuló háló számára generálok adathalmazt. A bemutatott szoftver segítségével lehetőség nyílik az adathalmaz előállítására és az átalakítások szemléltetésére.

A szakdolgozatomban az adathalmaz generálására használt, Python kódokat átalakító szoftvert mutatom be. A szoftver célja minél több ekvivalens átalakítás megvalósítása, az átalakítások szemléltetéséhez grafikus felhasználói felület is készül. Az átalakítások lehetnek egyszerűbbek (pl. azonosítók átnevezése), vagy összetettebbek (pl. a listát előállító for-ciklus "comprehension” kifejezéssé alakítása). Ezek az átalakítások a forráskód absztrakt szintaxisfájának módosításával valósíthatóak meg.
