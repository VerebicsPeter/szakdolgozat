\chapter{Bevezetés}
\label{ch:intro}

Szakdolgozatom témája Python forráskódok átalakítása, és ezen átalakítások szemléltetése.
A motiváció az átalakítások mögött egy olyan adathalmaz generálása amiben ekvivalens és nem
ekvivalens 
forráskód-párok egyaránt szerepelnek. Egy ilyen adathalmazt felhasználhatunk egy mélytanuló
neuronháló tanítására,
ami forráskód-párok ekvivalenciáját dönti el.

Az ekvivalencia eldöntése fontos feladat, mivel egyre több, kódokat gépi tanulással refaktoráló
eszköz létezik.
Ezek az eszközök egy kódot változtatva sokszor a kód jelentését is megváltoztatják.
Egy ekivalenciát eldöntő neuronháló képes lenne kiszűrni az ilyen eszközök által generált
rossz eredményeket, javítva az eszközök hatékonyságán.
Tehát ekvivalens és nem ekvivalens kódokat generálva felépíthetünk egy adathalmazt, ami
ekvivalenciával felcímkézett kódpárokat tartalmaz, és alkalmas egy fent leírt neuronháló
tanítására.

Az általam implementált átalakítások absztrakt szintaxisfák (AST-k) módosításával működnek.
Egy forráskód fordítása alatt a szemantikus elemző előállítja a kód AST-jét,
ami a kódot egy fa adatstruktúrával reprezentálja.
Az AST-nek a szemantikus elemzésben van szerepe,
de használhatjuk kódok átalakítására is, mivel visszaakítható forráskóddá.

Az általam megvalósított átalakítások a Python \emph{ast} modult \cite{pythonAST} használják,
amely része a Python standard könyvtárának.

\pagebreak

Az átalakításokat szabályok végzik. Átalakításkor 
a Python kódból létrehozott AST-n végrehajthatunk egy szabályt.
A szabály megváltoztatja az AST-t, amit ha visszaalakítunk kóddá,
egy megváltozott Python kódot kapunk.

A szakdolgozatomban ekvivalens és nem ekvivalens szabályokat is definiálok.
Egy szabály akkor tekinthető ekvivalensnek, ha a kód szemantikáját nem változtatja meg.
Például a Python-ban is teljesül a valós számok körében a szorzás kommutatív tulajdonsága.
Tehát, ha egy Python kódban két szám szorzásánál a bal és jobb operandust megcseréljük,
akkor a szorzás eredménye nem változik, vagyis ez az átalakítás ekvivalens.
Ez a példa természetesen nagyon egyszerű, a szakdolgozatomban összetettebb átalakítások
is szerepelnek.

Az adathalmazban az ekvivalens kódok generálásához saját szabályok mellett
a \emph{ruff} Python linter és formatter szabályait is felhasználtam.
A \emph{ruff} már létező Python lintereket implementál Rust programozási nyelven,
így sok más Python refaktoráló eszköz szabályait is képes elvégezni,
amlyek tökéletesek az általam implementált szabályok kiegészítésére.

A szakdolgozatom következő fejezeteiben az adathalmaz generálására és az átalakítások
szemléltetésére alkalmas szoftver használatát és működését részletezem.
